% Options for packages loaded elsewhere
\PassOptionsToPackage{unicode}{hyperref}
\PassOptionsToPackage{hyphens}{url}
\PassOptionsToPackage{dvipsnames,svgnames,x11names}{xcolor}
%
\documentclass[
  letterpaper,
  DIV=11,
  numbers=noendperiod]{scrreprt}

\usepackage{amsmath,amssymb}
\usepackage{iftex}
\ifPDFTeX
  \usepackage[T1]{fontenc}
  \usepackage[utf8]{inputenc}
  \usepackage{textcomp} % provide euro and other symbols
\else % if luatex or xetex
  \usepackage{unicode-math}
  \defaultfontfeatures{Scale=MatchLowercase}
  \defaultfontfeatures[\rmfamily]{Ligatures=TeX,Scale=1}
\fi
\usepackage{lmodern}
\ifPDFTeX\else  
    % xetex/luatex font selection
\fi
% Use upquote if available, for straight quotes in verbatim environments
\IfFileExists{upquote.sty}{\usepackage{upquote}}{}
\IfFileExists{microtype.sty}{% use microtype if available
  \usepackage[]{microtype}
  \UseMicrotypeSet[protrusion]{basicmath} % disable protrusion for tt fonts
}{}
\makeatletter
\@ifundefined{KOMAClassName}{% if non-KOMA class
  \IfFileExists{parskip.sty}{%
    \usepackage{parskip}
  }{% else
    \setlength{\parindent}{0pt}
    \setlength{\parskip}{6pt plus 2pt minus 1pt}}
}{% if KOMA class
  \KOMAoptions{parskip=half}}
\makeatother
\usepackage{xcolor}
\setlength{\emergencystretch}{3em} % prevent overfull lines
\setcounter{secnumdepth}{5}
% Make \paragraph and \subparagraph free-standing
\ifx\paragraph\undefined\else
  \let\oldparagraph\paragraph
  \renewcommand{\paragraph}[1]{\oldparagraph{#1}\mbox{}}
\fi
\ifx\subparagraph\undefined\else
  \let\oldsubparagraph\subparagraph
  \renewcommand{\subparagraph}[1]{\oldsubparagraph{#1}\mbox{}}
\fi


\providecommand{\tightlist}{%
  \setlength{\itemsep}{0pt}\setlength{\parskip}{0pt}}\usepackage{longtable,booktabs,array}
\usepackage{calc} % for calculating minipage widths
% Correct order of tables after \paragraph or \subparagraph
\usepackage{etoolbox}
\makeatletter
\patchcmd\longtable{\par}{\if@noskipsec\mbox{}\fi\par}{}{}
\makeatother
% Allow footnotes in longtable head/foot
\IfFileExists{footnotehyper.sty}{\usepackage{footnotehyper}}{\usepackage{footnote}}
\makesavenoteenv{longtable}
\usepackage{graphicx}
\makeatletter
\def\maxwidth{\ifdim\Gin@nat@width>\linewidth\linewidth\else\Gin@nat@width\fi}
\def\maxheight{\ifdim\Gin@nat@height>\textheight\textheight\else\Gin@nat@height\fi}
\makeatother
% Scale images if necessary, so that they will not overflow the page
% margins by default, and it is still possible to overwrite the defaults
% using explicit options in \includegraphics[width, height, ...]{}
\setkeys{Gin}{width=\maxwidth,height=\maxheight,keepaspectratio}
% Set default figure placement to htbp
\makeatletter
\def\fps@figure{htbp}
\makeatother
% definitions for citeproc citations
\NewDocumentCommand\citeproctext{}{}
\NewDocumentCommand\citeproc{mm}{%
  \begingroup\def\citeproctext{#2}\cite{#1}\endgroup}
\makeatletter
 % allow citations to break across lines
 \let\@cite@ofmt\@firstofone
 % avoid brackets around text for \cite:
 \def\@biblabel#1{}
 \def\@cite#1#2{{#1\if@tempswa , #2\fi}}
\makeatother
\newlength{\cslhangindent}
\setlength{\cslhangindent}{1.5em}
\newlength{\csllabelwidth}
\setlength{\csllabelwidth}{3em}
\newenvironment{CSLReferences}[2] % #1 hanging-indent, #2 entry-spacing
 {\begin{list}{}{%
  \setlength{\itemindent}{0pt}
  \setlength{\leftmargin}{0pt}
  \setlength{\parsep}{0pt}
  % turn on hanging indent if param 1 is 1
  \ifodd #1
   \setlength{\leftmargin}{\cslhangindent}
   \setlength{\itemindent}{-1\cslhangindent}
  \fi
  % set entry spacing
  \setlength{\itemsep}{#2\baselineskip}}}
 {\end{list}}
\usepackage{calc}
\newcommand{\CSLBlock}[1]{\hfill\break\parbox[t]{\linewidth}{\strut\ignorespaces#1\strut}}
\newcommand{\CSLLeftMargin}[1]{\parbox[t]{\csllabelwidth}{\strut#1\strut}}
\newcommand{\CSLRightInline}[1]{\parbox[t]{\linewidth - \csllabelwidth}{\strut#1\strut}}
\newcommand{\CSLIndent}[1]{\hspace{\cslhangindent}#1}

\KOMAoption{captions}{tableheading}
\makeatletter
\@ifpackageloaded{tcolorbox}{}{\usepackage[skins,breakable]{tcolorbox}}
\@ifpackageloaded{fontawesome5}{}{\usepackage{fontawesome5}}
\definecolor{quarto-callout-color}{HTML}{909090}
\definecolor{quarto-callout-note-color}{HTML}{0758E5}
\definecolor{quarto-callout-important-color}{HTML}{CC1914}
\definecolor{quarto-callout-warning-color}{HTML}{EB9113}
\definecolor{quarto-callout-tip-color}{HTML}{00A047}
\definecolor{quarto-callout-caution-color}{HTML}{FC5300}
\definecolor{quarto-callout-color-frame}{HTML}{acacac}
\definecolor{quarto-callout-note-color-frame}{HTML}{4582ec}
\definecolor{quarto-callout-important-color-frame}{HTML}{d9534f}
\definecolor{quarto-callout-warning-color-frame}{HTML}{f0ad4e}
\definecolor{quarto-callout-tip-color-frame}{HTML}{02b875}
\definecolor{quarto-callout-caution-color-frame}{HTML}{fd7e14}
\makeatother
\makeatletter
\@ifpackageloaded{bookmark}{}{\usepackage{bookmark}}
\makeatother
\makeatletter
\@ifpackageloaded{caption}{}{\usepackage{caption}}
\AtBeginDocument{%
\ifdefined\contentsname
  \renewcommand*\contentsname{Table of contents}
\else
  \newcommand\contentsname{Table of contents}
\fi
\ifdefined\listfigurename
  \renewcommand*\listfigurename{List of Figures}
\else
  \newcommand\listfigurename{List of Figures}
\fi
\ifdefined\listtablename
  \renewcommand*\listtablename{List of Tables}
\else
  \newcommand\listtablename{List of Tables}
\fi
\ifdefined\figurename
  \renewcommand*\figurename{Figure}
\else
  \newcommand\figurename{Figure}
\fi
\ifdefined\tablename
  \renewcommand*\tablename{Table}
\else
  \newcommand\tablename{Table}
\fi
}
\@ifpackageloaded{float}{}{\usepackage{float}}
\floatstyle{ruled}
\@ifundefined{c@chapter}{\newfloat{codelisting}{h}{lop}}{\newfloat{codelisting}{h}{lop}[chapter]}
\floatname{codelisting}{Listing}
\newcommand*\listoflistings{\listof{codelisting}{List of Listings}}
\makeatother
\makeatletter
\makeatother
\makeatletter
\@ifpackageloaded{caption}{}{\usepackage{caption}}
\@ifpackageloaded{subcaption}{}{\usepackage{subcaption}}
\makeatother
\ifLuaTeX
  \usepackage{selnolig}  % disable illegal ligatures
\fi
\usepackage{bookmark}

\IfFileExists{xurl.sty}{\usepackage{xurl}}{} % add URL line breaks if available
\urlstyle{same} % disable monospaced font for URLs
\hypersetup{
  pdftitle={Textbook},
  pdfauthor={Norah Jones},
  colorlinks=true,
  linkcolor={blue},
  filecolor={Maroon},
  citecolor={Blue},
  urlcolor={Blue},
  pdfcreator={LaTeX via pandoc}}

\title{Textbook}
\author{Norah Jones}
\date{2024-07-29}

\begin{document}
\maketitle

\renewcommand*\contentsname{Table of contents}
{
\hypersetup{linkcolor=}
\setcounter{tocdepth}{2}
\tableofcontents
}
\bookmarksetup{startatroot}

\chapter*{Preface}\label{preface}
\addcontentsline{toc}{chapter}{Preface}

\markboth{Preface}{Preface}

This is a Quarto book.

To learn more about Quarto books visit
\url{https://quarto.org/docs/books}.

\bookmarksetup{startatroot}

\chapter{Intro to data visualization and graphical
grammars}\label{intro-to-data-visualization-and-graphical-grammars}

\begin{tcolorbox}[enhanced jigsaw, title=\textcolor{quarto-callout-note-color}{\faInfo}\hspace{0.5em}{Learning outcomes}, colframe=quarto-callout-note-color-frame, colbacktitle=quarto-callout-note-color!10!white, titlerule=0mm, left=2mm, coltitle=black, toprule=.15mm, colback=white, opacityback=0, bottomtitle=1mm, rightrule=.15mm, bottomrule=.15mm, leftrule=.75mm, arc=.35mm, toptitle=1mm, breakable, opacitybacktitle=0.6]

\begin{itemize}
\item
  \textbf{Explain}: The main advantages of visualizing data instead of
  presenting it with numbers is that they are easier to interpret for
  humans.
\item
  \textbf{Describe}: A grammar of graphics defines the grammatical rules
  that can be used to construct entire visualizations from smaller
  building blocks.
\item
  \textbf{Apply}: Be able to use the visualization grammars Altair and
  ggplot and create a basic chart via \texttt{alt.Chart()...} /
  \texttt{ggplot()...}
\item
  \textbf{Explain}: The advantage of high level level syntax allows us
  to think in terms of the data, rather than graphical details.
\end{itemize}

\end{tcolorbox}

\subsection{What is data
visualization?}\label{what-is-data-visualization}

At its core, data visualization is about representing numbers with
graphical elements such as the position of a line, the length of a bar,
or the colour of a point. We often use visualizations to explore data
ourselves, and to effectively communicate our insights to others, as we
will learn in later modules of this course.

\subsection{What is the purpose of visualizing
data?}\label{what-is-the-purpose-of-visualizing-data}

We often visualize data in order to help us answer a specific question
we have about our dataset, but it can also help us generate new
questions.

Before creating a visualization, it is important that you think about
why you are making it, and what you want to achieve from creating this
plot. Is there a specific question you are trying to answer, like
comparing the relationship between two dataframe columns? Or are you
creating a plot to help you understand the structure of your data more
in general, such as plotting the distribution of each dataframe column?

In either case, it can be extremely helpful to draw out your plot with
pen and paper first. This helps you think about if the plot you are
creating makes sense or if there is another plot better suited for the
task at hand. Drawing with pen and paper also makes it easier to write
the code afterwards, since you clearly know what you are expecting the
visualization to look like.

\subsection{Why bother visualizing data instead of showing raw
numbers?}\label{why-bother-visualizing-data-instead-of-showing-raw-numbers}

To understand why visualizations are so powerful, it is helpful to
remember that to answer a question, we often have to put the data in a
format that is easy for us humans to interpret. Because our number
systems have only been around for about 5,000 years, we need to assert
effort and train ourselves to recognize structure in numerical data.

Visual systems, on the other hand, have undergone refinement during
500,000,000 years of evolution, so we can instinctively recognize visual
patterns and accurately estimate visual properties such as colours and
distances.

Practically, this means that we can arrive at correct conclusions faster
from studying visual rather than numerical representations of the same
data. For example, have a look at the four sets of numbers in the table
on the slide. Can you see the differences in the general trends between
these four sets of numbers? This is a slightly modified version of the
original, which was put together by statistician Francis Anscombe in the
70s.

\begin{longtable}[]{@{}ll@{}}
\caption{}\label{T_1ae4d}\tabularnewline
\toprule\noalign{}
\multicolumn{2}{@{}l@{}}{%
A} \\
X & Y \\
\midrule\noalign{}
\endfirsthead
\toprule\noalign{}
\multicolumn{2}{@{}l@{}}{%
A} \\
X & Y \\
\midrule\noalign{}
\endhead
\bottomrule\noalign{}
\endlastfoot
10.00 & 8.04 \\
8.00 & 6.95 \\
13.00 & 7.58 \\
9.00 & 8.81 \\
11.00 & 8.33 \\
14.00 & 9.96 \\
6.00 & 7.24 \\
4.00 & 4.26 \\
12.00 & 10.84 \\
7.00 & 4.81 \\
5.00 & 5.68 \\
\end{longtable}

\begin{longtable}[]{@{}ll@{}}
\caption{}\label{T_519d6}\tabularnewline
\toprule\noalign{}
\multicolumn{2}{@{}l@{}}{%
B} \\
X & Y \\
\midrule\noalign{}
\endfirsthead
\toprule\noalign{}
\multicolumn{2}{@{}l@{}}{%
B} \\
X & Y \\
\midrule\noalign{}
\endhead
\bottomrule\noalign{}
\endlastfoot
10.00 & 9.14 \\
8.00 & 8.14 \\
13.00 & 8.74 \\
9.00 & 8.77 \\
11.00 & 9.26 \\
14.00 & 8.10 \\
6.00 & 6.13 \\
4.00 & 3.10 \\
12.00 & 9.13 \\
7.00 & 7.26 \\
5.00 & 4.74 \\
\end{longtable}

\begin{longtable}[]{@{}ll@{}}
\caption{}\label{T_a5387}\tabularnewline
\toprule\noalign{}
\multicolumn{2}{@{}l@{}}{%
C} \\
X & Y \\
\midrule\noalign{}
\endfirsthead
\toprule\noalign{}
\multicolumn{2}{@{}l@{}}{%
C} \\
X & Y \\
\midrule\noalign{}
\endhead
\bottomrule\noalign{}
\endlastfoot
10.00 & 7.46 \\
8.00 & 6.77 \\
13.00 & 8.50 \\
9.00 & 7.11 \\
11.00 & 7.81 \\
14.00 & 8.84 \\
6.00 & 6.08 \\
4.00 & 5.39 \\
12.00 & 8.15 \\
7.00 & 6.42 \\
5.00 & 5.73 \\
\end{longtable}

\begin{longtable}[]{@{}ll@{}}
\caption{}\label{T_4cad9}\tabularnewline
\toprule\noalign{}
\multicolumn{2}{@{}l@{}}{%
D} \\
X & Y \\
\midrule\noalign{}
\endfirsthead
\toprule\noalign{}
\multicolumn{2}{@{}l@{}}{%
D} \\
X & Y \\
\midrule\noalign{}
\endhead
\bottomrule\noalign{}
\endlastfoot
8.00 & 6.58 \\
8.00 & 5.76 \\
8.00 & 7.71 \\
8.00 & 8.84 \\
8.00 & 8.47 \\
8.00 & 7.04 \\
8.00 & 5.25 \\
19.00 & 12.50 \\
8.00 & 5.56 \\
8.00 & 7.91 \\
8.00 & 6.89 \\
\end{longtable}

\subsubsection{Summary statistics don't tell the whole
story}\label{summary-statistics-dont-tell-the-whole-story}

You are likely not able to see much difference between the data sets in
the table above. What about if I showed you a few commonly used
numerical summaries of the data?

\begin{longtable}[]{@{}ll@{}}
\caption{}\label{T_40ada}\tabularnewline
\toprule\noalign{}
\multicolumn{2}{@{}l@{}}{%
A} \\
X & Y \\
\midrule\noalign{}
\endfirsthead
\toprule\noalign{}
\multicolumn{2}{@{}l@{}}{%
A} \\
X & Y \\
\midrule\noalign{}
\endhead
\bottomrule\noalign{}
\endlastfoot
9.00 & 7.50 \\
3.32 & 2.03 \\
\end{longtable}

\begin{longtable}[]{@{}ll@{}}
\caption{}\label{T_ff4fb}\tabularnewline
\toprule\noalign{}
\multicolumn{2}{@{}l@{}}{%
B} \\
X & Y \\
\midrule\noalign{}
\endfirsthead
\toprule\noalign{}
\multicolumn{2}{@{}l@{}}{%
B} \\
X & Y \\
\midrule\noalign{}
\endhead
\bottomrule\noalign{}
\endlastfoot
9.00 & 7.50 \\
3.32 & 2.03 \\
\end{longtable}

\begin{longtable}[]{@{}ll@{}}
\caption{}\label{T_33ae9}\tabularnewline
\toprule\noalign{}
\multicolumn{2}{@{}l@{}}{%
C} \\
X & Y \\
\midrule\noalign{}
\endfirsthead
\toprule\noalign{}
\multicolumn{2}{@{}l@{}}{%
C} \\
X & Y \\
\midrule\noalign{}
\endhead
\bottomrule\noalign{}
\endlastfoot
9.00 & \textbf{7.11} \\
3.32 & \textbf{1.15} \\
\end{longtable}

\begin{longtable}[]{@{}ll@{}}
\caption{}\label{T_f567f}\tabularnewline
\toprule\noalign{}
\multicolumn{2}{@{}l@{}}{%
D} \\
X & Y \\
\midrule\noalign{}
\endfirsthead
\toprule\noalign{}
\multicolumn{2}{@{}l@{}}{%
D} \\
X & Y \\
\midrule\noalign{}
\endhead
\bottomrule\noalign{}
\endlastfoot
9.00 & 7.50 \\
3.32 & 2.03 \\
\end{longtable}

Summaries, such as the mean and standard deviation, are helpful
statistical tools that are often useful for detecting the differences
between datasets. However, since they collapse the data into just a few
numbers, statistical summaries can't tell the whole story about the data
and there can be important differences between datasets that summaries
fail to reveal.

Below, the mean and standard deviation indicate that set C is slightly
different from the other sets of data in terms of the centre of the
sample distribution and the spread of that distribution, while the
remaining three sets of data have a similar centre and spread.

\subsubsection{Plotting the data immediately reveals patterns in the
data}\label{plotting-the-data-immediately-reveals-patterns-in-the-data}

So if you can't really see any patterns in the data and the statistical
summaries are the same, that must mean that the four sets are pretty
similar, right? Sounds about right to me so let's go ahead and plot them
to have a quick look and\ldots{}

\begin{verbatim}
alt.FacetChart(...)
\end{verbatim}

\ldots{} what the\ldots{} how\ldots{} there must be something wrong,
right? Well what is wrong is that humans are not good at detecting
patterns in raw numbers, and we don't have good intuition about which
combination of numbers can contribute to the same statistical summaries.
But guess what we excel at? Detecting visual patterns!

It is immediately clear to us how these sets of numbers differ once they
are shown as graphical objects instead of textual objects. We could not
detect these patterns from only looking at the raw numbers or summary
statistics This is one of the main reasons why data visualization is
such a powerful tool for data exploration and communication.

In our example here, we would come to widely different conclusions about
the behaviour of the data for the four different data sets.

Sets A and C are roughly linearly increasing at similar rates, whereas
set B reaches a plateau and starts to drop, and set D has a constant
X-value for all numbers except one big outlier.

\bookmarksetup{startatroot}

\chapter{Summary}\label{summary}

In summary, this book has no content whatsoever.

\bookmarksetup{startatroot}

\chapter*{References}\label{references}
\addcontentsline{toc}{chapter}{References}

\markboth{References}{References}

\phantomsection\label{refs}
\begin{CSLReferences}{0}{1}
\end{CSLReferences}



\end{document}
